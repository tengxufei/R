%% LyX 2.0.6 created this file.  For more info, see http://www.lyx.org/.
%% Do not edit unless you really know what you are doing.
\documentclass{article}\usepackage[]{graphicx}\usepackage[]{color}
%% maxwidth is the original width if it is less than linewidth
%% otherwise use linewidth (to make sure the graphics do not exceed the margin)
\makeatletter
\def\maxwidth{ %
  \ifdim\Gin@nat@width>\linewidth
    \linewidth
  \else
    \Gin@nat@width
  \fi
}
\makeatother

\definecolor{fgcolor}{rgb}{0.345, 0.345, 0.345}
\newcommand{\hlnum}[1]{\textcolor[rgb]{0.686,0.059,0.569}{#1}}%
\newcommand{\hlstr}[1]{\textcolor[rgb]{0.192,0.494,0.8}{#1}}%
\newcommand{\hlcom}[1]{\textcolor[rgb]{0.678,0.584,0.686}{\textit{#1}}}%
\newcommand{\hlopt}[1]{\textcolor[rgb]{0,0,0}{#1}}%
\newcommand{\hlstd}[1]{\textcolor[rgb]{0.345,0.345,0.345}{#1}}%
\newcommand{\hlkwa}[1]{\textcolor[rgb]{0.161,0.373,0.58}{\textbf{#1}}}%
\newcommand{\hlkwb}[1]{\textcolor[rgb]{0.69,0.353,0.396}{#1}}%
\newcommand{\hlkwc}[1]{\textcolor[rgb]{0.333,0.667,0.333}{#1}}%
\newcommand{\hlkwd}[1]{\textcolor[rgb]{0.737,0.353,0.396}{\textbf{#1}}}%

\usepackage{framed}
\makeatletter
\newenvironment{kframe}{%
 \def\at@end@of@kframe{}%
 \ifinner\ifhmode%
  \def\at@end@of@kframe{\end{minipage}}%
  \begin{minipage}{\columnwidth}%
 \fi\fi%
 \def\FrameCommand##1{\hskip\@totalleftmargin \hskip-\fboxsep
 \colorbox{shadecolor}{##1}\hskip-\fboxsep
     % There is no \\@totalrightmargin, so:
     \hskip-\linewidth \hskip-\@totalleftmargin \hskip\columnwidth}%
 \MakeFramed {\advance\hsize-\width
   \@totalleftmargin\z@ \linewidth\hsize
   \@setminipage}}%
 {\par\unskip\endMakeFramed%
 \at@end@of@kframe}
\makeatother

\definecolor{shadecolor}{rgb}{.97, .97, .97}
\definecolor{messagecolor}{rgb}{0, 0, 0}
\definecolor{warningcolor}{rgb}{1, 0, 1}
\definecolor{errorcolor}{rgb}{1, 0, 0}
\newenvironment{knitrout}{}{} % an empty environment to be redefined in TeX

\usepackage{alltt}
\usepackage[sc]{mathpazo}
\usepackage[T1]{fontenc}
\usepackage{geometry}
\geometry{verbose,tmargin=2.5cm,bmargin=2.5cm,lmargin=2.5cm,rmargin=2.5cm}
\usepackage{url}
\usepackage[unicode=true,pdfusetitle,
 bookmarks=true,bookmarksnumbered=true,bookmarksopen=true,bookmarksopenlevel=2,
 breaklinks=false,pdfborder={0 0 1},backref=false,colorlinks=false]
 {hyperref}
\hypersetup{
 pdfstartview={XYZ null null 1}}
\usepackage{listings}
\lstloadlanguages{Python}
\usepackage{pgfplotstable}
\usepackage{csvsimple}
\IfFileExists{upquote.sty}{\usepackage{upquote}}{}


\begin{document}
\definecolor{keywords}{RGB}{255,0,90}
\definecolor{comments}{RGB}{0,0,113}
\definecolor{red}{RGB}{160,0,0}
\definecolor{green}{RGB}{0,150,0}
 
 
\lstset{framextopmargin=50pt}
\lstset{language=Python, 
        basicstyle=\ttfamily\small, 
        keywordstyle=\color{keywords},
        commentstyle=\color{comments},
        stringstyle=\color{red},
        showstringspaces=false,
        identifierstyle=\color{green},
        breaklines=true}
        
\renewcommand\thesection{\arabic{section}}

%%% for TOC with numbering

\title{Trifolium Pallescens genome annotation}


\author{Vikas Gupta and Stig U. Andersen}

\maketitle
\setcounter{secnumdepth}{5}
\setcounter{tocdepth}{5}
\tableofcontents
\newpage


% add dot after number
\makeatletter
\g@addto@macro\thesection.
\makeatother




\section{Data}

\begin{knitrout}
\definecolor{shadecolor}{rgb}{0.969, 0.969, 0.969}\color{fgcolor}\begin{kframe}
\begin{alltt}
p <- \hlkwd{path.expand}(\hlstr{"/Volumes/GenomeDK/LotusGenome/01_vgupta/11_clover/20140709_clover_pallescens/RogerMoraga"})
\hlkwd{setwd}(p)
p
\end{alltt}
\begin{verbatim}
## [1] "/Volumes/GenomeDK/LotusGenome/01_vgupta/11_clover/20140709_clover_pallescens/RogerMoraga"
\end{verbatim}
\begin{alltt}
\hlcom{## File size}
a <- \hlkwd{system}(\hlstr{"ls -lh | awk \hlstr{'\{print $9,$5\}'}"}, intern = TRUE)

\hlkwd{for} (i in a) \{
    \hlkwd{print}(i)
\}
\end{alltt}
\begin{verbatim}
## [1] " "
## [1] "F1_consensus.fasta.gz 54M"
## [1] "L4_consensus.fasta.gz 34M"
## [1] "T_pal_stolon_1_1_9_1.fastq.gz 3.8G"
## [1] "T_pal_stolon_1_1_9_2.fastq.gz 3.9G"
\end{verbatim}
\end{kframe}
\end{knitrout}


\end{document}
